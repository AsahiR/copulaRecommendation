本研究では,コピュラを用いた既存の推薦システムの問題点を改善することで,ユーザのもつより複雑な嗜好を汲み取れるような推薦システムの構築を目指す.\par
情報推薦システムは,膨大な情報の中からユーザの嗜好にあった情報を提示するシステムである.
情報推薦システムの代表的アルゴリズムの一つであるコンテンツベースフィルタリングでは,ユーザが好むアイテムをから嗜好モデルを構築し,それに基づいて推薦を行う.\par
コンテンツベースフィルタリングで既存の推薦システムの一つに,コピュラという確率モデルを用いて嗜好モデルを構築するものがある.これは,ユーザが関心を示したアイテムを教師データとして嗜好モデルを構築するものであり,機械学習手法よりも学習結果の分析が容易であるというメリットをもつ.
しかし,このシステムの問題点として特徴量の分布が正規分布であると仮定している点,離散値の特徴量が扱えない点,特異な特徴量分布に対応できない点,などがある.\par
本論文では,これらの問題点に対し,カーネル密度推定や,ノイズを考慮した関心度式や,許容範囲フィルターを用いた手法などを組み合わせた手法を提案する.
評価実験の結果,既存の手法で用いられたデータセットでは,提案手法が既存手法と同等の結果を示し,離散値特徴量を追加した新たなデータセットでは,通常のケースと,アイテムが2値の離散値をとる特徴量をもつ場合や特徴量の分布が前述の特異な分布になるような場合を含めて,提案手法が有意水準1\%以下で既存手法よりも結果が良いことが確認できた.
