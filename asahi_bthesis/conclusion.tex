\chapter{結論}
\hspace{1em}本論文では,既存の鈴木らの手法の問題点を解決するために主にカーネル密度推定や,ノイズを除去した関心度式や,許容範囲フィルターを用いた手法を提案し,その評価を行った.\par
鈴木らのシステムの問題点として,特徴量の分布に正規分布を仮定している点や,離散値の特徴量を扱えない点や,特異な分布に対応できない点を指摘した.
指摘した問題点の解決手法として,正規分布モデルをカーネル密度推定に変更する手法や,離散値特徴量と連続値特徴量の関心度を適切に比較できるような新たな関心度式$Att_{Shr}$を用いる手法や,特異な分布に対応するために許容範囲フィルターを用いる手法などを提案した.\par
評価実験の結果,本提案手法は既存手法と比較して,旧データセットでは同程度の性能を示し,指摘した問題ケースを含む新たなデータセットでは,有意水準1\%以下でより高い性能を示すことを確認できた.
%unfinished
特に新たなデータセットでは,提案手法が比較手法の中でも機械学習手法である$SVM$よりも高い性能を示すことを確認できた.
よって,提案手法は通常の嗜好ケースで既存手法と同程度の性能を維持しつつ,特異な嗜好に対して既存手法や$SVM$のような機械学習手法よりも高い性能を示すといえる.\par
本研究の課題点としては,離散値のマッピング手法に離散値が2値という前提のもとで式\eqref{eq:ppl}で定義される人気度を用いたが,3値以上の離散値でも有効な手法を考案すべき点と,カーネル密度推定で用いるパラメータであるバンド幅$h$の選択手法についてより適切な手法の存在の有無の調査をすべき点と,許容範囲フィルターをより適切に用いるために特異な分布をとる特徴量をより高精度に抽出する方法を研究すべき点などがある.
