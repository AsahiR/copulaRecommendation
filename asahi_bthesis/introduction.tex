%\chapter{Introduction}
\chapter{序論}
\hspace{1em}近年,インターネットが普及しweb技術が進化することで,日々大量の情報が発信されている.
一方で,大量の情報の中から人々が自らにとって有益な情報を選択することは困難になっている.
このような問題を解決するために,ユーザの嗜好を汲み取りユーザ自身に適したアイテムを推薦するための情報推薦技術が研究され,注目を集めている.\par
情報推薦の代表的なアルゴリズムには,協調フィルタリング\cite{user-based-collaborative-filtering}\cite{item-based-collaborative-filtering}とコンテンツベースフィルタリング\cite{content-based-filtering}がある.
前者はユーザの評価履歴から求めたユーザ同士の類似度を用いて推薦を行うもので,後者はアイテムが持つ特徴量とユーザの嗜好情報から推薦を行うものである.
後者は特徴量ベースで推薦を行うため,前者と比較して新しいアイテムを推薦する場合や推薦システムの利用者が少ない場合でも機能するというメリットがある.
そこで,本研究はコンテンツベースフィルタリングを用いる.\par
コンテンツベースフィルタリングには,ユーザの評価履歴を教師データとした学習ベースで嗜好モデルを構築する手法がある.
学習ベースの手法は,ユーザ自身が嗜好情報を回答することで嗜好モデルを構築するような手法と比べて,ユーザ自身が把握しきれないような嗜好を汲み取れたり,ユーザへの負担が小さいというメリットがあるため,本研究では学習ベースのコンテンツベースフィルタリングを扱う.\par
学習手法には,機械学習手法\cite{svm}\cite{neural-network}や確率モデル\cite{Suzuki}を用いるものがある.機械学習手法には学習結果の解釈が容易でないという問題が存在するのに対して,情報推薦分野においては学習結果を解釈し,そこから新たな知見を得ることには大きな意味がある.\par
そこで鈴木ら\cite{Suzuki}は,結果の解釈が容易である確率モデルのコピュラを用いた学習手法を提案した.
鈴木らのシステムは,コピュラに特徴量の累積分布を入力することで特徴量を統合するものである.
また,鈴木らは特徴量毎にユーザが持つ関心度が異なることに着目している.
鈴木らのシステムはこの関心度とコピュラに基づいて推薦を行うため,高い精度で嗜好モデルを構築できる.\par
しかし,鈴木らのシステムにはいくつかの問題点がある.
第一に特徴量の分布モデルに正規分布を仮定しているため,実際の特徴量の分布が多峰の分布の場合に推薦精度が落ちる可能性がある.
第二に離散値の特徴量を扱えない点がある.鈴木らのシステムは特徴量の累積分布を利用するが,特徴量が離散値の場合,累積分布を定義できないため,離散値の特等量が扱えない.
第三に特異な分布,例えば特徴量値の範囲にこだわりをもつような場合,その特徴量の分布を正しく扱えないという問題がある.
鈴木らのシステムでは,特徴量の累積分布をコピュラに入力したスコア値が累積分布の単調増加になることを利用して,スコア値の高いものを優先的に推薦する.
しかし,特徴量は,特徴を数値化したものなので,その数値が高ければ良いとは限らない.よって,特徴量の数値にこだわりを持つケース,例えば高低の両端に関心をもつが,それ以外の区間には関心をもたないケースや,逆に両端ではなく,ある特定の区間に関心をもつようなケースの場合,その特徴量分布を適切に扱うことできない問題が生じる.\par
こういった問題に対処するこで,ユーザのより複雑な嗜好を汲み取れるような推薦システムの構築が期待できる.
よって,本研究では特徴量の分布モデルにカーネル密度推定を採用し,
特異な分布をもつ特徴量に対しては,特徴量のスコア値から,ユーザが強い嗜好を示す範囲を抽出するフィルターを定義し,これを用いてより複雑な嗜好を汲み取れるような推薦手法を提案する.
%unfinished
さらに評価実験により提案手法が有効であることを示す.
